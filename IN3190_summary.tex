\documentclass{article}
\usepackage[utf8]{inputenc}
\usepackage{amsmath,graphicx,varioref,verbatim,amsfonts,geometry}
\usepackage[hidelinks]{hyperref}
\usepackage{listings}
\usepackage[T1]{fontenc}
\usepackage{mathtools}
\usepackage{textcomp}
\usepackage[english]{babel}
\usepackage{wrapfig}
\usepackage{fancyhdr}
\usepackage{amssymb}
\usepackage{enumitem}
\usepackage{xcolor}
\usepackage{lastpage}
\usepackage{graphicx}
\usepackage{matlab-prettifier}
\usepackage{filecontents}
\usepackage{underscore}
\usepackage{makecell}
\usepackage{caption}
\usepackage{subcaption}
\usepackage{siunitx}
\usepackage{mathrsfs}

\pagestyle{fancy}


%% Double underline 
\def\dubline#1{\underline{\underline{#1}}}

%% Matrix
\newcommand\SmallMatrix[1]{{%
		\arraycolsep=0.3\arraycolsep\ensuremath{\begin{pmatrix}#1\end{pmatrix}}}}
	
%% Codeinsert
\definecolor{listinggray}{gray}{0.9}
\definecolor{lbcolor}{rgb}{0.9,0.9,0.9}
\lstset{
	backgroundcolor=\color{lbcolor},
	tabsize=4,
	rulecolor=,
	language=python,
	basicstyle=\scriptsize,
	upquote=true,
	aboveskip={1.5\baselineskip},
	columns=fixed,
	numbers=left,
	showstringspaces=false,      
	extendedchars=true,
	breaklines=true,
	prebreak = \raisebox{0ex}[0ex][0ex]{\ensuremath{\hookleftarrow}},
	frame=single,
	showtabs=false,
	showspaces=false,
	showstringspaces=false,
	identifierstyle=\ttfamily,
	keywordstyle=\color[rgb]{0,0,1},
	commentstyle=\color[rgb]{0.133,0.545,0.133},
	stringstyle=\color[rgb]{0.627,0.126,0.941}
}
\newcounter{subproject}
\renewcommand{\thesubproject}{\alph{subproject}}
\newenvironment{subproj}{
	\begin{description}
		\item[\refstepcounter{subproject}(\thesubproject)]
	}{\end{description}}

\begin{document}
\title{IN3190 - Exam Preparation Questions H23\\
{\Large University of Oslo}}
\author{Daniel Tran}
\date{November 2023}
\fancyhead[L]{Should know to exam H23 - IN3190}
\fancyhead[C]{UiO}
\fancyhead[R]{Daniel Tran}
\cfoot{\thepage\ of \pageref{LastPage}}
\maketitle
\hypersetup{linkcolor=black}
\tableofcontents
\clearpage

\section{Essential Topics}
\subsection{Discrete time}
\subsubsection{Sine, cosine and exponential functions}

    Mathematical formula that establishes the fundamental relationship between the trigonometric functions and the complex exponential function. \newline
    Euler's formula states that for any real number $x$:
    \begin{equation}
        e^{ix} = \cos(x) + i\sin(x)
    \end{equation}
    When $x = \pi$, Euler's formula yields Euler's identity:
    \begin{equation}
        e^{i\pi} + 1 = 0
    \end{equation}
\subsubsection{Elementary discrete signals}
\paragraph{Unit impulse} %Spør gruppelærer om plot fra forelesning
Also known as the dirac delta function.
\begin{equation}
    \delta[n] = \begin{cases}
        1, & n = 0 \\
        0, & n \neq 0
    \end{cases}
    \hspace*{20pt}\text{may also be written as:}
    \hspace*{20pt} \delta[n] = u[n] - u[n-1]
\end{equation}

\begin{figure}[h!]
    \centering
    \includegraphics[width=0.5\textwidth]{figures/Discrete time/unit_impulse.png}
    \caption{Unit impulse}
    \label{fig:unit_impulse}
\end{figure}

\paragraph{Step function}
Also known as unit step, unit step function or heaviside function. The value which is zero for negative arguments and one for positive arguments.
\begin{equation}
    u[n] = \begin{cases}
        1, & n \geq 0 \\
        0, & n < 0
    \end{cases}
    \hspace*{20pt}\text{may also be written as:}
    \hspace*{20pt} u[n] = \sum_{k=0}^{\infty} \delta[n-k]
\end{equation}
\begin{figure}[h!]
    \centering
    \includegraphics[width=0.5\textwidth]{figures/Discrete time/unit_step_function.png}
    \caption{Unit step function}
    \label{fig:unit_step}
\end{figure}

\newpage
\paragraph{Ramp function}
Also known as the unit-ramp or unit ramp function. Graph shaped like a ramp.
\begin{equation}
u_r[n] = \begin{cases}
    n, & n \geq 0 \\
    0, & n < 0
\end{cases}
\end{equation}
\begin{figure}[h!]
    \centering
    \includegraphics[width=0.5\textwidth]{figures/Discrete time/unit_ramp_function.png}
    \caption{Unit ramp function}
    \label{fig:unit_ramp}
\end{figure}

\paragraph{Periodic sequences} 
$x[n]$ is periodic if and only if $x[n] = x[n+N]$
\begin{itemize}
    \item \textbf{Fundamental period}: 
    \newline Smallest positive integer N which fulfills the relation above
    \item \textbf{Sinusoidal sequences}: 
    \newline $x[n] = A\cos(\omega_0n + \phi)$, where 
    \begin{itemize}
        \item $A$ is amplitude,
        \item $\omega_0$ is the angular frequency
        \item and $\phi$ is the phase of $x[n]$.
    \end{itemize} 
    $x[n]$ is periodic if and only if $\omega_0N = 2\pi k$, for N and k as positive integers.
\end{itemize}

\begin{figure}[h!]
    \centering
    \includegraphics[width=0.5\textwidth]{figures/Discrete time/sinusoidal_sequences.png}
    \caption{Sinusoidal sequences}
    \label{fig:Sinusoidal_sequence}
\end{figure}

\subsubsection{LTI systems, including characteristics via the transformation between input and output}
\paragraph{Linearity:} The relationship between the input $x(t)$ and the output $y(t)$, both being regarded as functions. If $a$ is a constant then the system output to $ax(t)$ is $ay(t)$. Linear system if and only if it $H\{\cdot\}$ is both additive and homogeneous, in other words: If it fulfills the superposition principle. That is:
\begin{equation}
    H\{ax_1(t)+bx_x(t)\} = aH\{x_1(t)\} + bH\{x_2(t)\}
\end{equation}

\paragraph{Time-invariance:} 
A linear system is time-invariant or shift-invariant means that wether we apply an input to the system now or T seconds from now, the output will be identical except for a time delay of S seconds. In other words, if $y(t)$ is the output of a system with a input $x(t)$, then the output of the system with input $x(t-T)$ is $y(t-T)$. The system is invariant becasue the output does not depend on the particular time at which the input is applied.
\\
\\
The fundamental result in LTI system theory is that any LTI system can be characterized entirely by a single function called the system's impulse response $h(t)$. The output of the system $y(t)$ is simply the convolution of the input to the system $x(t)$ with the systems's impulse response $h(t)$.
\\
\\
LTI system can also be characterized in the frequency domain by the system's transfer function $H(s)$, which is the Laplace transform of the system's impulse response $h(t)$. Or Z-transform in the case of discrete time systems.
\begin{figure}[h!]
    \centering
    \includegraphics[width=0.5\textwidth]{figures/Discrete time/LTI_time_freq_domain.png}
    \caption{Relationship between the time domain and the frequency domain}
    \label{fig:lti_relationship}
\end{figure}

\subsection{Z-transform}

\subsection{Frequency analysis and DFT}

\subsection{Filter design}

\subsubsection{Advantages/Disadvantages of FIR and IIR filters}
\paragraph{Advantages, FIR filters:}
\begin{itemize}
    \item Can have exact linear phase, which makes h(n) zero points symmetric about $|z| = 1$
    \item  Filter structures are always stable for quantized coefficients.
    \begin{itemize}
        \item Stable IIR filters can become unstable due to rounding errors in coefficients.
    \end{itemize}
    \item Design methods are generally linear. 
    \item Start transient has finite length.
    \item Can be realized efficently in HW (but not as efficient as IIR filters)
\end{itemize}

\paragraph{Disadvantages, FIR filters:}


Generally require a higher order than IIR filters to meet the same specifications. This often ensure the FIR filter having greater calculation burden/complexity and larger group delay than the corresponding IIR filter.

\subsubsection{Common ways to design IIR filters}
\begin{itemize}
    \item Specify the filter as a analog/continuous time (CT) equivalent to the dicrete IIR filter you want to design.
    \item Design the analog lowpass filter in continuous time.
    \item Convert the analog lowpass filter to a digital filter using filter transformation.
    \item Convert the digital lowpass filter to the desired filter using a frequency band transformation.
    \item Convert the digital filter to a discrete time (DT) filter (From s-domain to z-domain).
    \item You could also just use an optimization algorithm (e.g. least-square method)
\end{itemize}

\paragraph{Note on standard approximations for IIR filter design:}
\begin{itemize}
    \item Conversion of the digital filter specification to an anolog low-pass filter specification
    \item Determination of analog low-pass transfer function $H_a(s)$
    \item Transformation of $H_a(s)$ to the desired digital transfer function $G(z)$
    This method is preferred because:
    \begin{itemize}
        \item Approximation of an analog transfer function is well established and suitable.
        \item Closed expressions for the analog approximation are often obtained.
        \item Large tables for analog filter design are available.
        \item In many cases, digital simulation of an analogue system is sought, i.e. the analogue specification exist.
    \end{itemize}
\end{itemize}

\subsubsection{Typical IIR filters}

\subsubsection{Common ways to design FIR filters}

\subsubsection{Window functions and the consequences of applying them}

\subsubsection{Linear phase and linear-phase filters}

\subsubsection{Grapihcal representation of filter structures and operations}

\subsubsection{Gibb's phenomenon}

\subsection{Sampling and reconstruction}
\subsubsection{Sampling theorem}
The sampling frequency $\Omega_T = 2\Omega_H$ is called the Nyquist frequency. If this term is satisfied, then the original signal can be recovered from the sampled signal. In practice we often use oversampling (sampling at a higher rate than the Nyquist frequency) to get an appropiate reconstruction.
\subsubsection{Synthesizing an arbitrary signal as a sum of unit impulses}

Sampling corresponds to multiplying with a Dirac comb with distance $T$ between the unit impulses.
In the Fourier domain, this corresponds to convolving with a Dirac comb with distance $F_T$ between
scaled unit impulses. (The distance is $\Omega_T$ if we look at the angular frequency axis)
\begin{figure}[h!]
    \centering
    \includegraphics[width=1\textwidth]{figures/Sampling and reconstruction/Sampling_dirac_impulses.png}
    \caption{Sampling with dirac impulses}
    \label{fig:sampling_dirac_impulses}
\end{figure}
\begin{equation}
    x_p (t) = x_a (t) III_T (t) = \sum_{n=-\infty}^{\infty} x_a (t) \delta (t - nT) = \sum_{n=-\infty}^{\infty} x_a (nT) \delta (t - nT) \triangleq x(n)
\end{equation}

\clearpage
\subsubsection{Sampling of band-limited signals}
\begin{figure}[h!]
    \centering
    \includegraphics[width=0.75\textwidth]{figures/Sampling and reconstruction/sampling_band_lim_signal.png}
    \caption{Periodic sampling in freq. domain}
    \label{fig:sampling_band_lim_signal}
\end{figure}
The reconstructed signal will repeat itself in the frequency domain. As long the sampling frequency ($\Omega_T$)is equal or greater than the Nyquist rate ($2\Omega_H$), we will be able to seperate the original signal from the aliases. The reconstruction in fig. \ref{fig:sampling_band_lim_signal} can be written as a convolution:
\begin{equation}
    x_r (t) = \sum_{n=-\infty}^{\infty} x_a (nT) g_r (t - nT) 
\end{equation}

Where $g_r (t)$ is the interpolation reconstruction function (I.e. impulse response). The 

\subsubsection{Ideal reconstruction}

The fundamental copy of the Fourier-domain representation of the sampled signal can be found by the 'ideal' filter for the convolution operation is: 
\begin{equation}
    G_r (j\Omega) = \bigg\{ \begin{matrix}
        T, & |\Omega| \leq \Omega_T /2 \\
        0, & |\Omega| > \Omega_T / 2
    \end{matrix}
\end{equation}

This gives the ideal: $X_r (j\Omega) = X_a (j\Omega)$ \\ (Reconstructed Fourier domain representation equals the original Fourier domain representation).

\clearpage
\subsubsection{Upsampling and downsampling}
\begin{figure}[h!]
    \centering
    \includegraphics[width=1\textwidth]{figures/Sampling and reconstruction/sampling_examples.png}
    \caption{Example of over/undersampling}
    \label{fig:sampling_example}
\end{figure}

\section{Additional Topics}
\subsection{Discrete time}
\subsubsection{Symmetrical signals}

\subsection{LTI systems and characteristics}

\subsection{Convolution and correlation}

\subsection{Z-transform}

\subsection{Frequency analysis and DFT}

\subsection{Filter design}

\subsection{Sampling and reconstruction}
\end{document}